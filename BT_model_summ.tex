\documentclass{article}
\usepackage{amsmath, amssymb, graphicx, setspace, multirow,multicol,units,float,fullpage}

\doublespacing

\newcommand{\mathsym}[1]{{}}
\newcommand{\unicode}[1]{{}}

\newcounter{mathematicapage}

\graphicspath{{C:/Users/a.quinones/Dropbox/LosAndes/investigacion/BreedingTime}}


\begin{document}
	
	\title{The evolution of social effects in breeding time} \author{Andr\'es Qui\~nones} \date{December 2019}
	\maketitle
	
	\section{Justification}
	 
	 Individuals evolve the sensitivity to include in their decision making process of 
	 when to breed the reproductive status of their conspecifics. We start with an asexual
	 model. Where individuals can use an environmental cue to decide when during the year 
	 to breed. This environmental cue is partly correlated with the abundance of 
	 resources along the season. These resources are necessary to raise the offspring, 
	 and thus reproductive success (or offspring survival) depends on the per capita 
	 availability of resources. The perception of the environmental cue by each individual 
	 is prone to errors. So, the mismatch between the perceived cue, and the resource 
	 variation not predicted by the environmental cue, could potentially trigger individual
	 breeding at the wrong time. Hence, individuals could improve their decision making by 
	 taking into account the breeding status of their peers, which could potentially 
	 improve the accuracy of their decision by pulling together individual and social information. 
	 This flexibility comes with a potencial cost.  If all individuals respond to the social influence, breeding time might end up being overly concentrated in time, which, due to density dependence, might reduce the reproductive success. Thus, depending on the strength of the intra-specific competition,
	 it might be better for individuals to respond less to their peers. 
	 
	 The question of breeding time in a sexual population has some interesting complications.
	 The balance between information aquisition and intra-specific competition described
	 above applies to females only. For whom resources for offspring provisioning is of 
	 outmost importance. For males however, it's more important to find mates. Thus, 
	 environmental information is useful for males only as a proxy of female availability. 
	 Perhaps, the social influence is a more direct cue of the availability of reproductive 
	 females. This does not mean that males do not experience density dependence effects. 
	 If a male happens to be reproductively active when abunduce of females is low, but 
	 abundance of reproductively active males even lower, such male might enjoy a high 
	 reproductice success. Which opens up the question, what is the best strategy for a 
	 male to decide when to become reproductively active?
	 
	  
	 We aim to answer the above questions using a individual based model. In such model,
	 individuals have genotypes that define their response to a environmental cue, as well 
	 as the reproductive status of other individuals. The decision of whether to breed at a particular time is given the log odds equation from a logistic regression
	 \begin{equation}
		log\frac{p}{1-p}=\beta_0+
	 \end{equation}
	
	\bibliography{Cleanerlearning}
	\bibliographystyle{plain}
	
\end{document}